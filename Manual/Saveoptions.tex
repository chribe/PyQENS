\section{Plot and saving options}
There are several plot and saving options. The different options can be combined if they are handed in as a list.
\begin{itemize}
	\item \code{plotresults='fits'}: This option displays the fits. Fits of \Gls{qens} are displayed with $\hbar\omega$ as abscissa, while \gls{fws} are displayed with $q$ as abscissa. The ordinate in both cases is in logarithmic scale. In case \gls{fws} and \gls{qens} are combined, they are displayed similar to the pure \gls{qens} spectra. the \gls{fws} are plotted also in the plots. 
	\item \code{plotresults='results'}: This option displays only the fit results but not the Fits itself.
	\item \code{plotresults='all'}: This option combines both options of \code{plotresults='fits'} and \code{plotresults='results'}.
	\item \code{plotresults='silent'}: These option generates the same figures as \code{plotresults='all'} but they are not displayed but only saved.
	\item \code{plotresults='none'}: This option does not create any figures.
	\item \code{plotresults='writeascii'}: This option saves all fit results as ascii, which might be usefull, if they should be treated later with other programs.
\end{itemize}
Next to plotting the fit results as explained above, a summary can be generated by activating the option \code{writesummary='all/fit/results/none'}. The options \code{writesummary='all/fit/results'} overwrites the option \code{plotresults='none'} to \code{plotresults='all/fit/results'}.

Files are saved as default as ".eps" files. To change this can be saved by setting the flag \code{pictureextension} from \code{pictureextension='eps'} to \code{pictureextension='png'}